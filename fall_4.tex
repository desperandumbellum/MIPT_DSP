%\chapter*{Неделя 4}
\protect\thispagestyle{fancy}
\section{}
Используется $16$-разрядный биполярный АЦП и фильтр Баттерворта третьего порядка с частотой среза $f_c = 10$ КHz и амплитудно частотной характеристикой:
\[|\Capit{H}(f)| = \dfrac{1}{\sqrt{1 + (f/f_c)^6}}.\]

Предположим, что входной сигнал является широкополосным.

Оценить минимальное затухание в полосе подавления $\Capit{A}_{\min}$ для фильтра защиты от наложения.

Оценить минимальную частоту дискретизации $f_d$.

Используя формулу для относительного шума квантования, получим:
\begin{align*}
	\Capit{A}_{\min} = -\gamma_{16} = (6.02n + 1.76)\Big|_{n=16} = 98.08 \text{ dB}.
\end{align*}

\begin{equation*}
	\dfrac{1}{\Capit{A}^2_{\min}} = \dfrac{1}{1 + \left(f_{\max}/f_c\right)^6}, \quad \Rightarrow
	f_{\max} = f_c \sqrt[6]{\Capit{A}^2_{\min} - 1} \approx f_c \cdot \Capit{A}^{1/3}_{\min} = 10^{\frac{98.08}{3 \cdot 20}} \cdot f_c \approx 43.12 f_c \approx 431.2 \text{ KHz}.
\end{equation*}
\begin{equation*}
	f_d = 2f_{\max} = 862.2 \text{ KHz}.
\end{equation*}

\section{}
Аналоговый фильтр Чебышёва I рода $3$-го порядка с частотой среза $f_c = 40$ Hz имеет величину пульсаций $0.5$ dB. Определить ослабление, вносимое этим фильтром на частоте $2f_c$.

\begin{align*}
	&|\Capit{H}(\nu_c)|^2 = 10 \lg \dfrac{1}{1 + \varepsilon^2 \Capit{T}^2_3 (\nu_c)} = -0.5 \text{ dB}, \quad \Rightarrow
	\varepsilon^2 = \dfrac{10^{0.5/10} - 1}{\Capit{T}^2_3(\nu_c)} = 10^{0.05} - 1 = 0.122018.\\
	&|\Capit{H}(2\nu_c)|^2 = 10 \lg \dfrac{1}{1 + \varepsilon^2 \Capit{T}^2_3 (2\nu_c)}\footnotemark = 
	10 \lg \dfrac{1}{1 + \varepsilon^2 [4(2\nu_c)^3 - 3(2\nu_c)]^2} =
	10 \lg \dfrac{1}{1 + 0.122018 \cdot 26^2} = -19.216 \text{ dB}.
\end{align*}
\footnotetext{$\Capit{T}_3(\nu) = 2\nu \Capit{T}_2(\nu) - \Capit{T}_1 (\nu)  =  2\nu[2\nu \Capit{T}_1(\nu) - \Capit{T}_0 (\nu)] - \Capit{T}_1 (\nu) = 4\nu^3 - 3\nu$.}

\section{}
Вещественный сигнал $x(t)$ с полосой $2f_b = 10$ KHz ($f_b$ -- верхняя граничная частота) дискретизуется с минимально возможной частотой дискретизации в соответствии с теоремой отсчётов. В результате получается последовательность $x(k)$. Обозначим через $\Capit{X}[n]$ $1000$-точечное ДПФ последовательности $x(k)$.

\begin{enumerate}[label=(\alph*)]
	\item Каким частотам (в Hz) в ДВПФ последовательности $x(k)$ соответствуют частоты ДПФ с номерами $n_1 = 100$ и $n_2 = 850$?
	
	\item Каким частотам (в Hz) в спектре исходного сигнала $x(t)$ соответствуют индексы $n_1 = 100$ и $n_2 = 850$ в последовательности $\Capit{X}[n]$?
\end{enumerate}

Частота дискретизации должна быть равна $f_d = 2f_b = 10$ KHz.

\begin{enumerate}[label=(\alph*)]
	\item Для дискретизованного сигнала:
	\begin{align*}
		f_1 = \dfrac{n_1 f_d}{N} = 1\text{ KHz}, \quad f_2 = \dfrac{n_2 f_d}{N} = 8.5\text{ KHz}.
	\end{align*}
	\item Для исходного аналогового сигнала:
	\begin{align*}
		f_1 = \dfrac{n_1 f_d}{N} = 1\text{ KHz}, \quad f_2 = \dfrac{(n_2 - N)f_d}{N} = -1.5\text{ KHz}.
	\end{align*}
\end{enumerate}
