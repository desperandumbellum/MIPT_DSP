%\chapter*{Неделя 6}
\protect\thispagestyle{fancy}
\section{}
Непрерывный стационарный случайный процесс обладает узкополосной плотностью мощности $G(f)$, равной нулю при $|f|\geq10$ кГц.
На интервале в $\Delta t = 10$ c реализация этого случайного процесса подвергается дискретизации с частотой
$f_d = 20$ кГц, после чего спектральная плотность мощности оценивается методом усредненных периодограмм (методом Бартлетта). При вычислении периодограмм используется разбиение последовательности на сегменты и вычисление ДПФ на каждом сегменте:

\begin{gather*}
\hat{G}_p(n\Delta f) = \dfrac{\Delta t}{D} |X_p[n]|^2,\\
X_p[n] = \sum\limits_{k=0}^{D-1}x_p(k\Delta t)\exp\left(-j\dfrac{2\pi}{N_{\text{FFT}}}nk\right),
\quad n=0, 1, \ldots, N_{\text{FFT}}-1,\\
x_p(k\Delta t) = x\left((pD+k)\Delta t\right),\quad p=0,1,\ldots, P-1.
\end{gather*}
Длина сегментов сигнала совпадает с размерностью ДПФ и равна $D$.


\begin{enumerate}[label=(\alph*)]
	\item Чему равна длина $N$ последовательности, по которой производится оценка?
	\item При каком наименьшем значении $D$ расстояние между частотами,	в которых вычисляется спектральная оценка, не превышает $\Delta f_{max} = 10$ Гц?
	\item Какое число неперекрывающихся сегментов $P$ используется при анализе в предположении, что их длина соответствует результату предыдущего пункта?
	\item Нам хотелось бы уменьшить дисперсию оценки в $2$ раза, сохранив	расстояние между отсчетами ДПФ по оси частот. Сформулируйте метод достижения поставленной цели.
\end{enumerate}

\vspace{1cm}
\begin{enumerate}[label=(\alph*)]
	\item Общая длина последовательности равна $N = \Delta t \cdot f_d = 200000$.
	\item $\Delta f = \dfrac{f_d}{N_{\text{FFT}}} = \dfrac{f_d}{D}$. Чтобы выполнялось $\Delta f \leq 10$ Hz, достаточно выбрать $D \geq \dfrac{f_d}{\Delta f_{max}} = 2000$.
	\item Поскольку сегменты не перекрываются, то $P = N/D = 100$.
	\item Чтобы уменьшить дисперсию оценки в $2$ раза, нужно увеличить число сегментов $P$ по которым осуществляется оценка. Это можно сделать, например, в 2 раза увеличив длительность исходной последовательности $\Delta t$ или использовав перекрывающиеся сегменты, что позволить увеличить $P$ без изменений входной последовательности.
\end{enumerate}

